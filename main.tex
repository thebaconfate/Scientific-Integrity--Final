\documentclass[a4paper, 14pt]{extarticle}
\usepackage{newtxtext}
 \usepackage{newtxmath}

% Load the VUB package.
% This has many options, please read the documentation at
% https://gitlab.com/rubdos/texlive-vub

\usepackage{hyperref}

% Some highly suggested packages, please read their manuals.
\usepackage{cleveref}
\usepackage[natbib,style=apa]{biblatex}
\addbibresource{bibliography.bib}
\makeatletter
\renewcommand{\maketitle}{\bgroup\setlength{\parindent}{2mm}
\begin{flushleft}
  \textbf{\@author}\\
  \vspace{1mm}
  \underline{\@title}
\end{flushleft}\egroup
}
\makeatother
\title{The human price for automation}
\date{}% Note, no
\author{Faculty of Sciences and Bioengineering Sciences: Computer Science \\ \vspace{1mm} Gérard Lichtert}
\begin{document}
\maketitle
\begin{flushleft}
Has your job been taken over by a robot or a program yet? From cashiers to warehouse operators, even drivers or artists are being replaced by either robots or AI. But is this a good thing? Or are we just blindly replacing people with machines and programs? What about the people who lost their jobs?
\end{flushleft}
\begin{flushleft}
Due to technological advancements, for example, smarter robots, self-checkouts, chatbots, etc., companies are replacing their human workforce, leading to technological unemployment. According to \cite{mdpi}, that summarises several studies that have been dedicated to estimating the impact of automation on jobs. Depending on the location, 35–85\% of the workforce has a high risk ($>70\%$) of being automated. To make things worse, losing your job may be the least of your worries.
\end{flushleft}
\begin{flushleft}
Automation should be controlled. Or, at the very least, dealt with properly. While you might say, "But we have seen automation before; think of the industrial revolution!". Well, yes, while you might be right, it is not at the same scale as the current automation trends. \cite{mdpi} states that while the Industrial Revolution indeed led to technological unemployment, it also created jobs in other sectors, thus balancing the job market. However, this industrial revolution is different; it is more advanced and thus causes a lot more unemployment than it creates jobs.
\end{flushleft}
\begin{flushleft}
The less educated people in our society would be the first to go. As their tasks are usually the easiest to automate. But what can those folks do that cannot be automated? They would be doomed to be unemployed unless they got a higher education, which not everybody can afford or achieve in the first place.
\end{flushleft}
\begin{flushleft}
What would we do with all the free time we would have? While yes, I agree that we probably would have all the time to chase our dreams or hobbies at the beginning, \cite{mdpi} states that it would become a very real problem for us to find purpose in all that free time and that without a job to give us purpose, it could imply dire psychological consequences, such as idleness, boredom, or even depression.
\end{flushleft}
\begin{flushleft}
You could say that the rate at which humans will be replaced by robots or programs might be overestimated, as \cite{sciencedaily} states. And that replacement is not really what is happening. "Rather, workplaces are integrating both employees and robots in ways that generate more value for human labour." However, \cite{PDizikes2020} clearly states that while the rate at which humans might be replaced is overestimated, it is happening. Dizikes, interestingly, also adds another point to the discussion that automation causes inequality in wages because employees now also have to compete against robots.
\end{flushleft}
\begin{flushleft}
The philosopher Laura Maguire in \cite{philo} is concerned with how humans will survive without a job because, while it may involve more free time, people still need to pay bills. She suggests a universal basic income for everybody to survive (UBI); however, the only country that has tried this (and failed according to the New York Times) so far is Finland. Laura adds that it probably is not an option before finalizing that it probably is not everybody's cup of tea to be unemployed with a lot of free time.
\end{flushleft}
So we know that automation has serious consequences, but it is unavoidable. So, how do we deal with automation? \cite{mdpi} suggests some solutions but splits them into two categories. One is to mitigate the causes of technological unemployment, and the other is aimed at dealing with technological unemployment.
\begin{flushleft}
To mitigate technological unemployment, one could enhance workers instead of replacing them. Another measure could be to share work, reducing the weekly work hours, which could provide some benefits.
\end{flushleft}
\begin{flushleft}
To deal with unemployment, most solutions cover either re-education, either by oneself or by the company. Redistribution of wealth through a reform of tax, or the UBI, or a combination of education and redistribution of wealth.
\end{flushleft}
\begin{flushleft}
We have explained the implications of the current technological advancements—not only jobs are in danger, but there are other consequences. We have also seen some proposed solutions. However, will these be enough? Will they be implemented in time? Something must be done before the robots completely take over.
\end{flushleft}
\printbibliography
\end{document}
